\section{Conclusion}
\label{sec:conclusion}

In this paper, we have presented the performance characteristics of data
transfers for GPU computing. 
We found that the hardware-assisted DMA and the I/O read-and-write
access methods are the most effective to maximize the data transfer
performance for a single stream, while the microcontroller-based method
can overlap data transfers with the DMA and the IO read-and-write access
methods reducing the total makespan of multiple data streams.
We also showed that the standard real-time CPU scheduler can
shield the performance of data transfers from competing workload.
They are novel findings and useful contributions to the development of
low-latency real-time systems integrating GPUs.
The implementations of the investigated data transfer methods are
provided in the Gdev project at \url{http://github.com/cs005/gdev/}.

In future work, we will investigate how to determine the choice of data
transfer methods depending on the target system and workload.
Since data transfers are abstracted by the API in terms of user
programs, the runtime system must understand environments and choose
appropriate data transfer methods to meet the performance and latency
requirements of workload.
