\section{Related Work}
\label{sec:related_work}

The zero-copy data transfer methods for low-latency GPU computing were
developed for plasma control systems~\cite{Kato_ICCPS13}.
The authors argued that the hardware-based DMA transfer method does not
meet the latency requirement of plasma control systems.
They presented several zero-copy data transfer methods, some of which is
similar to the memory-mapped I/O read and write method investigated in
this paper.
However, this previous work considered only a small size of data.
This specific assumption allowed the I/O read and write method to
perform always better than the hardware-based DMA method.
We demonstrated that these two methods outperform each other depending
on the target data size.
In this regard, we provided more general observations of data transfer
methods for GPU computing.

The performance boundary of the hardware-based DMA and the I/O read and
write methods was briefly discussed in the Gdev
project~\cite{Kato_ATC12}. 
They showed that the hardware-based DMA method should be used only for
large data.
We provided the same claim in this paper.
However, we dig into the causal relation of these two methods more in
depth and also expanded our attention to the microcontroller-based
method.
Our findings complement the results of the Gdev project.

The scheduling of GPU data transfers was presented to improve the
responsiveness of GPU computing~\cite{Basaran_ECRTS12, Kato_RTSS11}.
These work focused on making preemption points for burst non-preemptive
DMA transfers with the GPU, but the underlying system relied on the
proprietary closed-source software.
On the other hand, we provided open-source implementations to disclose
the fundamental of GPU data transfer methods.
We found that the hardware-based DMA transfer method is not necessarily
the best choice depending on the data size and workload.
Since the I/O read and write method is fully preemptive and the
microcontroller-based method is partly preemptive, the contribution of
this paper provides a new insight to these preemptive data transfer
approaches.